% Digital Logic Report Template
% Created: 2020-01-10, John Miller

%==========================================================
%=========== Document Setup  ==============================

% Formatting defined by class file
\documentclass[11pt]{article}

% ---- Document formatting ----
\usepackage[margin=1in]{geometry}	% Narrower margins
\usepackage{booktabs}				% Nice formatting of tables
\usepackage{graphicx}				% Ability to include graphics

%\setlength\parindent{0pt}	% Do not indent first line of paragraphs 
\usepackage[parfill]{parskip}		% Line space b/w paragraphs
%	parfill option prevents last line of pgrph from being fully justified

% Parskip package adds too much space around titles, fix with this
\RequirePackage{titlesec}
\titlespacing\section{0pt}{8pt plus 4pt minus 2pt}{3pt plus 2pt minus 2pt}
\titlespacing\subsection{0pt}{4pt plus 4pt minus 2pt}{-2pt plus 2pt minus 2pt}
\titlespacing\subsubsection{0pt}{2pt plus 4pt minus 2pt}{-6pt plus 2pt minus 2pt}

% ---- Hyperlinks ----
\usepackage[colorlinks=true,urlcolor=blue]{hyperref}	% For URL's. Automatically links internal references.

% ---- Code listings ----
\usepackage{listings} 					% Nice code layout and inclusion
\usepackage[usenames,dvipsnames]{xcolor}	% Colors (needs to be defined before using colors)

% Define custom colors for listings
\definecolor{listinggray}{gray}{0.98}		% Listings background color
\definecolor{rulegray}{gray}{0.7}			% Listings rule/frame color

% Style for Verilog
\lstdefinestyle{Verilog}{
	language=Verilog,					% Verilog
	backgroundcolor=\color{listinggray},	% light gray background
	rulecolor=\color{blue}, 			% blue frame lines
	frame=tb,							% lines above & below
	linewidth=\columnwidth, 			% set line width
	basicstyle=\small\ttfamily,	% basic font style that is used for the code	
	breaklines=true, 					% allow breaking across columns/pages
	tabsize=3,							% set tab size
	commentstyle=\color{gray},	% comments in italic 
	stringstyle=\upshape,				% strings are printed in normal font
	showspaces=false,					% don't underscore spaces
}

% How to use: \Verilog[listing_options]{file}
\newcommand{\Verilog}[2][]{%
	\lstinputlisting[style=Verilog,#1]{#2}
}

\begin{document}

\title{ELC 2137 Lab 06: MUX and 7-segment decoder}
\author{My Nguyen}

\maketitle

\section*{Summary}
This lab's purpose is to set up a circuit to display an 8-bit number on two 7-segment displays using Verilog. First, create a multiplexer to switch between the two displays, each display will use 4 bits. Secondly, create a seven-segment decoder to covert the output from the multiplexer into hexadecimal digit for the 7-segment LED display. Then, a top-level module is created to wrap around the multiplexer and decoder proving input and output. Finally, import a constraint file for Basys3 board and create a bitstream file for Basys3 board. 

\section*{Table and Figure}
%table and simulation in a single figure environment
\begin{figure}[ht]
	\centering
	
	\includegraphics[width=\textwidth,trim=37cm 30cm 1cm 8cm,clip]{"mux2_4b"}
	\caption{Multiplexer ERT}
	\includegraphics[width=\textwidth,trim=35cm 30cm 1cm 8cm,clip]{"decoder"}
	\caption{7-bit Decoder ERT}
	\includegraphics[width=\textwidth,trim=37cm 25cm 1cm 8cm,clip]{"sseg1_test"}
	\caption{sseg1 ERT}
\end{figure}

\section*{Code}
\Verilog[firstline=23, caption=Multiplexer Implementation]{../verilog_source/mux2_4b.sv}
\Verilog[firstline=23,caption=Multiplexer Test Bench]{../verilog_source/mux2_4b_test.sv}
\Verilog[firstline=23,caption=7-bit Decoder Implementation]{../verilog_source/sseg_decoder.sv}
\Verilog[firstline=23,caption=7-bit Decoder Test Bench]{../verilog_source/sseg_decoder_test.sv}
\Verilog[firstline=23,caption=Decoder Wrap Implementation]{../verilog_source/wrap.sv}
\Verilog[firstline=23,caption=Decoder Wrap Test Bench]{../verilog_source/sseg1_test.sv}

\section*{Screenshot}
\begin{figure}[ht]
	\centering
	\includegraphics[width=12cm]{"board1"}
	\caption{First digit}
	\includegraphics[width=12cm]{"board2"}
	\caption{Second digit}
\end{figure}

\section*{Questions}
3. Most errors found where from typo and misunderstanding the prompt. A particular happens with created a bitstream file, where if you don't get the decoder wrap file as top, it will not causes an error telling you to fix your constrains file. \\
4. Nine wires are connected to the 7-segment display. If the segments were not all connected together, the decoder output will create a piority network instead of a parallel, which will make it harder to create segment display and harder to debug. 
\end{document}
