% Digital Logic Report Template
% Created: 2020-01-10, John Miller

%==========================================================
%=========== Document Setup  ==============================

% Formatting defined by class file
\documentclass[11pt]{article}

% ---- Document formatting ----
\usepackage[margin=1in]{geometry}	% Narrower margins
\usepackage{booktabs}				% Nice formatting of tables
\usepackage{graphicx}				% Ability to include graphics

%\setlength\parindent{0pt}	% Do not indent first line of paragraphs 
\usepackage[parfill]{parskip}		% Line space b/w paragraphs
%	parfill option prevents last line of pgrph from being fully justified

% Parskip package adds too much space around titles, fix with this
\RequirePackage{titlesec}
\titlespacing\section{0pt}{8pt plus 4pt minus 2pt}{3pt plus 2pt minus 2pt}
\titlespacing\subsection{0pt}{4pt plus 4pt minus 2pt}{-2pt plus 2pt minus 2pt}
\titlespacing\subsubsection{0pt}{2pt plus 4pt minus 2pt}{-6pt plus 2pt minus 2pt}

% ---- Hyperlinks ----
\usepackage[colorlinks=true,urlcolor=blue]{hyperref}	% For URL's. Automatically links internal references.

% ---- Code listings ----
\usepackage{listings} 					% Nice code layout and inclusion
\usepackage[usenames,dvipsnames]{xcolor}	% Colors (needs to be defined before using colors)

% Define custom colors for listings
\definecolor{listinggray}{gray}{0.98}		% Listings background color
\definecolor{rulegray}{gray}{0.7}			% Listings rule/frame color

% Style for Verilog
\lstdefinestyle{Verilog}{
	language=Verilog,					% Verilog
	backgroundcolor=\color{listinggray},	% light gray background
	rulecolor=\color{blue}, 			% blue frame lines
	frame=tb,							% lines above & below
	linewidth=\columnwidth, 			% set line width
	basicstyle=\small\ttfamily,	% basic font style that is used for the code	
	breaklines=true, 					% allow breaking across columns/pages
	tabsize=3,							% set tab size
	commentstyle=\color{gray},	% comments in italic 
	stringstyle=\upshape,				% strings are printed in normal font
	showspaces=false,					% don't underscore spaces
}

% How to use: \Verilog[listing_options]{file}
\newcommand{\Verilog}[2][]{%
	\lstinputlisting[style=Verilog,#1]{#2}
}

\begin{document}

\title{ELC 2137 Lab 05: Verilog Intro}
\author{My Nguyen}

\maketitle

\section*{Summary}
This lab purpose is to learn the basic Verilog syntax, organize files and folder structure to recreate half-adder, full-adder, and 2-bit adder/subtractor. First, familiarize by creating a RTL project using Basys3 board. From this create a half-adder file and populate it with code and logic for a half-adder, then create a test file to exhaustively test it. Repeat this process for full-adder and 2-bit adder/subtractor. 

\section*{Table and Figure}
%table and simulation in a single figure environment
\begin{figure}[ht]
	\centering
	
	\includegraphics[width=\textwidth,trim=40cm 30cm 2cm 8cm,clip]{"half_adder"}
	\caption{Half Adder ERT}
	\includegraphics[width=\textwidth,trim=40cm 30cm 2cm 8cm,clip]{"full_adder"}
	\caption{Full Adder ERT}
	\includegraphics[width=\textwidth,trim=40cm 25cm 2cm 8cm,clip]{"adder_subtractor"}
	\caption{2-bit Adder/Subtractor ERT}
\end{figure}

\section*{Code}
\Verilog[firstline=23, caption=Half Adder Implementation]{../verilog_code/half_adder.sv}
\Verilog[firstline=23,caption=Half Adder Test Bench]{../verilog_code/half_adder_test.v}
\Verilog[firstline=23,caption=Full Adder Implementation]{../verilog_code/full_adder.v}
\Verilog[firstline=23,caption=Full Adder Test Bench]{../verilog_code/full_adder_test.sv}
\Verilog[firstline=23,caption=2-bit Adder/Subtractor Implementation]{../verilog_code/adder_subtractor.sv}
\Verilog[firstline=23,caption=2-bit Adder/Subtractor Test Bench]{../verilog_code/adder_subtractor_test.sv}
\section*{Screenshot}
\begin{figure}[ht]
	\centering
	\includegraphics[width=\textwidth]{"lab5_block_diagram"}
	\caption{Block Diagrams}
\end{figure}

\section*{Questions}
4. The simulations matches exact the result from lab 3 and lab 4. \\
5. One thing I still cannot figure out is how to assign value to a multi-bit variable using another multi-bit variable.

\end{document}
