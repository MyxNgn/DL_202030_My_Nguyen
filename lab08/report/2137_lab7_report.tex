% Digital Logic Report Template
% Created: 2020-01-10, John Miller

%==========================================================
%=========== Document Setup  ==============================

% Formatting defined by class file
\documentclass[11pt]{article}

% ---- Document formatting ----
\usepackage[margin=1in]{geometry}	% Narrower margins
\usepackage{booktabs}				% Nice formatting of tables
\usepackage{graphicx}				% Ability to include graphics

%\setlength\parindent{0pt}	% Do not indent first line of paragraphs 
\usepackage[parfill]{parskip}		% Line space b/w paragraphs
%	parfill option prevents last line of pgrph from being fully justified

% Parskip package adds too much space around titles, fix with this
\RequirePackage{titlesec}
\titlespacing\section{0pt}{8pt plus 4pt minus 2pt}{3pt plus 2pt minus 2pt}
\titlespacing\subsection{0pt}{4pt plus 4pt minus 2pt}{-2pt plus 2pt minus 2pt}
\titlespacing\subsubsection{0pt}{2pt plus 4pt minus 2pt}{-6pt plus 2pt minus 2pt}

% ---- Hyperlinks ----
\usepackage[colorlinks=true,urlcolor=blue]{hyperref}	% For URL's. Automatically links internal references.

% ---- Code listings ----
\usepackage{listings} 					% Nice code layout and inclusion
\usepackage[usenames,dvipsnames]{xcolor}	% Colors (needs to be defined before using colors)

% Define custom colors for listings
\definecolor{listinggray}{gray}{0.98}		% Listings background color
\definecolor{rulegray}{gray}{0.7}			% Listings rule/frame color

% Style for Verilog
\lstdefinestyle{Verilog}{
	language=Verilog,					% Verilog
	backgroundcolor=\color{listinggray},	% light gray background
	rulecolor=\color{blue}, 			% blue frame lines
	frame=tb,							% lines above & below
	linewidth=\columnwidth, 			% set line width
	basicstyle=\small\ttfamily,	% basic font style that is used for the code	
	breaklines=true, 					% allow breaking across columns/pages
	tabsize=3,							% set tab size
	commentstyle=\color{gray},	% comments in italic 
	stringstyle=\upshape,				% strings are printed in normal font
	showspaces=false,					% don't underscore spaces
}

% How to use: \Verilog[listing_options]{file}
\newcommand{\Verilog}[2][]{%
	\lstinputlisting[style=Verilog,#1]{#2}
}

\begin{document}

\title{ELC 2137 Lab 07: Binary Coded Decimal}
\author{My Nguyen}

\maketitle

\section*{Summary}
This lab's purpose is to implement the double-dabble algorithm to convert hex values into binary coded decimal (BCD) using Verilog. The double-dabble algorithm, could be divided into to two parts shifting the code and adding to 3 to any binary number that is larger than 4. First, add3 file is created to detect any binary number larger than 4 and add 3 to it. Then, bcd6 and bcd11 modules were created to convert a 6-bit and 11-bit input to BCD using multiple add3 modules. Finally, create a top-level module named sseg1\_BCD which uses all 16 switches, bcd11 module, mux2\_4b and sseg\_decoder and output to the seven segment display.

\section*{Table and Figure}
\begin{figure}[ht]
	\centering
	\includegraphics[width=\textwidth,trim=24cm 23cm 0.5cm 5cm,clip]{"add3"}
	\caption{Add3 ERT}
	\includegraphics[width=\textwidth,trim=24cm 23cm 0.5cm 5cm,clip]{"bcd6"}
	\caption{6-bit BCD Convertor ERT}
	\includegraphics[width=\textwidth,trim=24cm 22cm 0.5cm 5cm,clip]{"bcd11"}
	\caption{11-bit BCD Convertor ERT}
\end{figure}

\section*{Code}
\Verilog[firstline=23,caption=Add3 Implementation]{../verilog_code/add3.sv}
\Verilog[firstline=23,caption=Add3 Test Bench]{../verilog_code/add3_test.sv}
\Verilog[firstline=23,caption=6-bit BCD Convertor Implementation]{../verilog_code/bcd6.sv}
\Verilog[firstline=23,caption=6-bit BCD Convertor Test Bench]{../verilog_code/bcd6_test.sv}
\Verilog[firstline=23,caption=11-bit BCD Convertor Implementation]{../verilog_code/bcd11.sv}
\Verilog[firstline=23,caption=11-bit BCD Convertor Test Bench]{../verilog_code/bcd11_test.sv}
\Verilog[firstline=23,caption=11-bit BCD Convertor Test Bench]{../verilog_code/sseg1_BCD.sv}

\section*{Screenshot}
\begin{figure}[ht]
	\centering
	\includegraphics[width=12cm]{"board1"}
	\caption{First digit}
\end{figure}
\begin{figure}[ht]
	\centering
	\includegraphics[width=12cm]{"board2"}
	\caption{Second digit}
	\includegraphics[width=12cm]{"schematic"}
	\caption{11-bit Convertor Schematic}
\end{figure}

%\section*{Questions}

\end{document}
